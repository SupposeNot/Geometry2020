% !TEX TS-program = pdflatexmk
\documentclass[12pt]{amsart}

%\usepackage[parfill]{parskip}    % Activate to begin paragraphs with an empty line rather than an indent

\usepackage[margin=1in]{geometry}

\usepackage{amsmath,amssymb,amsthm,latexsym,graphicx}
\usepackage[normalem]{ulem}
\usepackage{setspace} %used for doublespacing, etc.
\usepackage{hyperref}
\usepackage{cancel}
\usepackage[dvipsnames,usenames]{color}
\usepackage[all]{xy}
\usepackage{fancyhdr}
\pagestyle{fancy}
	\renewcommand{\headrulewidth}{0.5pt} % and the line
	\headsep=1cm
	
\DeclareGraphicsRule{.tif}{png}{.png}{`convert #1 `dirname #1`/`basename #1 .tif`.png}

%Some useful environments.
\newtheorem{theorem}{Theorem}
\newtheorem{corollary}[theorem]{Corollary}
\newtheorem{conjecture}[theorem]{Conjecture}
\newtheorem{lemma}[theorem]{Lemma}
\newtheorem{proposition}[theorem]{Proposition}
\newtheorem{definition}[theorem]{Definition}
\newtheorem{example}[theorem]{Example}
\newtheorem{axiom}{Axiom}
\theoremstyle{remark}
\newtheorem{remark}{Remark}
\newtheorem*{observation}{Observation}
\newtheorem*{basic notion}{Basic Notion}
\newtheorem{fact}{Fact}
\newtheorem*{exercise}{Exercise}%[section]

%For GeoGebra
\usepackage{pgf,tikz,pgfplots}
\pgfplotsset{compat=1.15}
\usepackage{mathrsfs}
\usetikzlibrary{arrows}
\newcommand{\degre}{\ensuremath{^\circ}}


%Some useful shortcuts for our favorite sets of numbers
%Note, you can use these WITHOUT entering math mode

\newcommand{\RR}{\ensuremath{\mathbb R}} 
\newcommand{\NN}{\ensuremath{\mathbb N}}
\newcommand{\ZZ}{\ensuremath{\mathbb Z}}
\newcommand{\QQ}{{\ensuremath\mathbb Q}}
\newcommand{\CC}{\ensuremath{\mathbb C}}
\newcommand{\EE}{{\ensuremath\mathbb E}}

%Some useful shortcuts for formatting lists
\newcommand{\bc}{\begin{center}}
\newcommand{\ec}{\end{center}}
\newcommand{\be}{\begin{enumerate}}
\newcommand{\ee}{\end{enumerate}}
\newcommand{\bi}{\begin{itemize}}
\newcommand{\ei}{\end{itemize}}

%Some useful shortcuts for formatting mathematical symbols
\newcommand{\ol}[1]{\overline{#1}}
\newcommand{\oimp}[1]{\overset{#1}{\iff}} %labeled iff symbol
\newcommand{\bv}[1]{\ensuremath{ \vec{\mathbf{#1}}} } %makes a vector.
\newcommand{\mc}[1]{\ensuremath{\mathcal{#1}}} %put something in caligraphic font
\newcommand{\bsl}[1]{\texttt{\symbol{92}{\em #1}}} %for backslashes.
\newcommand{\normale}{\trianglelefteq}
\newcommand{\normal}{\triangleleft}

%Commenting tools --- You can ignore these, but if you have a question about latex and send me your source file, I'll use them to explain stuff to you.
\newcommand{\mpg}[1]{\marginpar{ #1}} %to put comments in margins
\usepackage{soul}
\definecolor{highlight}{rgb}{1,0.6,0.6}
\sethlcolor{highlight}
\newcommand{\hlm}[1]{\colorbox{highlight}{$\displaystyle #1$}}
\newtheoremstyle{mycomment}{\topsep}{-0in}{\small \itshape \sffamily}{}{\small \itshape\sffamily}{:}{.5em}{}
\theoremstyle{mycomment}
\newtheorem*{acomment}{\color{BrickRed}{Comment}}
\newcommand{\com}[1]{{\color{OliveGreen}\begin{acomment}{#1} 
\end{acomment}\noindent}}
\newcommand{\red}[1]{{\color{BrickRed} #1}}
\newcommand{\blue}[1]{{\color{MidnightBlue}#1}}
\newcommand{\green}[1]{{\color{OliveGreen}#1}}
\newcommand{\mwrong}[2]{\red{\cancel{#1}}\green{#2}}
\newcommand{\wrong}[2]{\red{\sout{#1}}\green{#2}}
\definecolor{OliveGreen}{rgb}{.3,.5,.2}
\definecolor{MidnightBlue}{rgb}{.3,.4,.6}
\newcommand{\pts}[1]{\hfill\blue{{#1}/5}}

\chead{MATH F305: Geometry Summary Sheet}
\pagestyle{fancy}
%Modify these items:
%\rhead{\emph{Your Full Name Here}}
%\lhead{\emph{HW \#0 --- 3/14/15}}


\begin{document}

\thispagestyle{fancy}
\section{Collected Observations, Basic Notions and Facts}
\subsection{Absolute Geometry}
\begin{axiom} A straight line segment can be drawn between any two distinct points.\end{axiom} 
\begin{axiom} A straight line segment may be extended to a line.\end{axiom}
\begin{axiom} Given a line segment a circle may be drawn with the segment as radius and one end point at the center.\end{axiom}
\begin{axiom} All right angles are congruent.\end{axiom}
\subsection{Observations}
\begin{observation} Sets of parallel lines converge to a point in an image when they are not parallel to the picture plane. Lines parallel to the picture plane remain parallel.
\end{observation}
\begin{basic notion} We have two projection models, the artist and screen, and the camera obscura.\end{basic notion}
\begin{fact} Parallel lines may be understood as having the same direction vector, as being equidistant from each other at all points, or as not meeting.\end{fact}
\begin{observation} We know of three kinds of perspective drawings of a box... One point, two point, and three point. They are determined by the number of vanishing points the edges of the box have. \end{observation}
\begin{definition} A \emph{pencil of lines} is either...
\end{definition} 

We should remember the methods discussed in class for constructing boxes or letters using mathematical perspective methods. Anyone willing to write up their favorite way of completing the letter T?

\begin{observation} The possible images of a point are...

The possible images of a line are...

The possible images of a plane are ...
\end{observation}
\begin{observation} When thinking about the images of geometric objects it is important to DRAW the lines connecting the objects THROUGH the observation point TO the image plane. It is also important to be GENERIC in placing our objects (unless there is good reason not to).\end{observation}
\begin{fact} Top and side views are \emph{parallel} projections onto a plane. They are not projections \emph{through a point}.\end{fact}
\begin{definition} A \emph{ray} is...
\end{definition}
\begin{definition} A \emph{line segment} is...
\end{definition}
\begin{definition} A \emph{line} is...
\end{definition}
\section{Collected Results}
Euclidean geometry axioms:
\begin{description}
\item[Two points determine a line] A straight line segment can be drawn between two points.
\item[Line extension] A straight line segment may be extended to a line.
\item[Line segment determines a circle] Given a line segment a circle may be drawn with the segment as radius, and one endpoint at the center.
\item[Right angle congruence] All right angles are congruent.
\item[Playfair's Axiom] Given a line and a point in a plane, the point not incident with the line, there exists a unique line in the plane that is incident with the point but not incident with the line. That is, there is a uinique line in the plane that is parallel to the first line and that passes through the point.
\item[Measurement] 
\item[Sum of Lengths]
\item[Line Segment Copy]
\item[Transitivity of Congruence]
\item[SAS]
\item[SSS]
\end{description}
Some Euclidean geometry theorems:
\begin{description}
\item[Sum of Interior Angles] is two right angles.
\item[Vertical Opposite Angles] If a line is cut by a transversal, then the vertical angles are congruent.
\item[Alternate Interior Angles] If a pair of lines is cut by a transversal, then the alternate interior angles are congruent.
\item[Corresponding Angles - $\square$ 4.1.2] If a pair of parallel lines is cut by a transversal, the corresponding angles are congruent.
\item[Triangle Angle Sum] The sum of the interior angles of a triangle is a straight angle.
\item[ASA]
\item[SAA]
\item[SAS similarity] If $\triangle ABC$ and $\triangle DEF$ such that $\angle ABC\cong\angle DEF$ and $\frac{||AB||}{||DE||}=\frac{||BC||}{||EF||}$ then $\triangle ABC\sim\triangle DEF$.
\item[AAA$\iff\sim\triangle$]
\end{description}
\begin{proposition}[$\bigcirc$ 1.1] Suppose we have a viewpoint $A$, a picture plane $\omega$, and a line segment $\ol{AB}$ parallel to the picture plane of length $L$. Then if $d$ is the distance between $A$ and $\omega$, and $D$ the distance between $A$ and the plane containing $AB$, the length of the image $l$ in $\omega$ is $l=\frac{Ld}{D}$.
\end{proposition}
\begin{proposition}[$\square$ 2.1] The intersection of two pencils of lines as defined in \S 2 is at most one line.\end{proposition}
\begin{theorem}[Ceva] Let $\triangle ABC$ be a triangle, and let $D,E$ and $F$ be on the lines $BC, CA$ and $AB$ respectively such that lines $AD, BE$ and $FD$ are concurrent. Then $$\frac{|AF|}{|FB|}\cdot\frac{|BD|}{|DC|}\cdot\frac{|CE|}{|EA|}=1.$$
\end{theorem}
\begin{theorem}[Converse of Ceva's Theorem]
\end{theorem}
\begin{theorem}[Menelaus] Let $\triangle ABC$ be a triangle, and let a transversal line intersect sides $BC, CA$ and $AB$ at $D, E$ and $J$ respectively such that $D, E$ and $J$ are distinct from $A, B$ and $C$. Then $$\frac{|AJ|}{|JB|}\cdot\frac{|BD|}{|DC|}\cdot\frac{|CE|}{|EA|}=\ldots.$$
\end{theorem}
\begin{definition} Harmonic set...\end{definition}
\begin{theorem}[Harmonic Ratio] Let $A, F, B$, and $J$ be four points along a line, in that order. This set of points is a harmonic set if and only if $$\frac{|AF|}{|FB|}\cdot\frac{|BJ|}{|JA|}=-1.$$\end{theorem}
\begin{theorem}[Star Trek Lemma]
\end{theorem}
\begin{theorem}[Thale] If $A$ is a point on a circle and $B,C$ are the end points of a diameter, then the angle $\angle BAC$ is  a right angle.\end{theorem}
\begin{lemma}[Carpenter's Lemma]
\end{lemma}
 \end{document}
 \end

  